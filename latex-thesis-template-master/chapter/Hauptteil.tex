\section{Balanced Scorecard}

\subsection{Einleitung}

Die Brau Union Österreich AG (BUÖ) ist das größte Brauereiunternehmen Österreichs und bildet das operative Rückgrat des BBAG-Konzerns. In einem durch hohen Wettbewerbsdruck, zunehmende Marktkonzentration und steigende Anforderungen der Kapitalmärkte geprägten Umfeld gewinnt eine wertorientierte Unternehmensführung zunehmend an Bedeutung. Ziel des Managements ist die nachhaltige Steigerung des Unternehmenswertes unter gleichzeitiger Berücksichtigung zentraler Stakeholder-Interessen.
Die Balanced Scorecard stellt ein geeignetes Instrument dar, um strategische Ziele in operative Steuerungsgrößen zu überführen und damit eine wertorientierte Unternehmensführung ganzheitlich zu unterstützen.

\subsection{Strategische Ausgangslage der BUÖ}

Die BUÖ verfolgt eine konsequente Mehrmarkenstrategie, um unterschiedliche Konsumentenbedürfnisse in verschiedenen Preis- und Image-Segmenten abzudecken. Der Vertrieb erfolgt über die zentralen Kanäle Lebensmittelhandel und Gastronomie. Mit einem Marktanteil von über 50 \% nimmt das Unternehmen eine dominante Stellung im österreichischen Biermarkt ein.
Das Controlling ist dezentral organisiert und auf einem hohen Entwicklungsniveau. Es unterstützt das Management sowohl in operativen als auch in strategischen Fragestellungen. Die wertorientierte Steuerung basiert insbesondere auf Kennzahlen wie Return on Investment, Unternehmensergebnis und Free Cashflow. Vor diesem Hintergrund ist eine Übersetzung dieser strategischen Zielgrößen in operative Werttreiber erforderlich.

\subsection{Bedeutung der Balanced Scorecard für die BUÖ}

Die Balanced Scorecard ermöglicht es, finanzielle Zielgrößen mit nicht-finanziellen Erfolgsfaktoren zu verknüpfen. Gerade für ein markenorientiertes Unternehmen wie die BUÖ sind Kundenbindung, Produktqualität und Mitarbeiterkompetenz entscheidende Werttreiber.
Durch die Strukturierung in vier Perspektiven wird sichergestellt, dass nicht ausschließlich finanzielle Kennzahlen betrachtet werden, sondern auch jene Faktoren, die langfristig zur Wertschaffung beitragen. Damit unterstützt die Balanced Scorecard die strategische Steuerung im Sinne eines ganzheitlichen Managementansatzes.

\subsection{Erläuterung der Perspektiven}
\subsubsection{Finanzperspektive}

Die Finanzperspektive steht im Einklang mit dem obersten Konzernziel der Wertschaffung. Zentrale Kennzahlen sind der Return on Investment und der Free Cashflow, da diese die nachhaltige Rentabilität des eingesetzten Kapitals widerspiegeln. Maßnahmen zur Effizienzsteigerung und zum gezielten Kapitaleinsatz sollen sicherstellen, dass die Rendite dauerhaft über den Kapitalkosten liegt.

\subsubsection{Kundenperspektive}

Da letztlich der Kunde die finanziellen Mittel generiert, nimmt die Kundenperspektive eine zentrale Rolle ein. Die Sicherung der Marktführerschaft erfordert eine hohe Kundenzufriedenheit sowie starke Marken. Die Mehrmarkenstrategie der BUÖ wird gezielt eingesetzt, um unterschiedliche Marktsegmente erfolgreich zu bedienen und langfristige Kundenbeziehungen aufzubauen.

\subsubsection{Interne Prozessperspektive}

Effiziente interne Prozesse bilden die Grundlage für Wettbewerbsfähigkeit und Profitabilität. Die BUÖ nutzt Skaleneffekte und Synergien innerhalb des Konzerns, um Produktions- und Logistikkosten zu senken. Gleichzeitig besitzt die Sicherstellung einer konstant hohen Produktqualität hohe Priorität, da diese unmittelbar auf die Kundenzufriedenheit wirkt.

\subsubsection{Lern- und Entwicklungsperspektive}

Motivierte und qualifizierte Mitarbeiter sind eine zentrale Voraussetzung für den langfristigen Unternehmenserfolg. Die BUÖ setzt daher auf kontinuierliche Weiterbildung und eine starke Controlling-Kompetenz. Innovationsfähigkeit, etwa durch neue Produkte oder Verpackungskonzepte, trägt zusätzlich zur nachhaltigen Sicherung der Wettbewerbsposition bei.

\subsection{Ursache-Wirkungs-Zusammenhänge}

Die Balanced Scorecard der BUÖ folgt einer klaren Ursache-Wirkungs-Logik. Investitionen in Mitarbeiterqualifikation und Controlling-Kompetenz verbessern interne Prozesse. Effiziente Prozesse führen zu hoher Produktqualität und Kundenzufriedenheit, was wiederum Absatz, Marktanteil und letztlich den Unternehmenswert steigert.

\subsubsection{Kritische Würdigung}

Die Balanced Scorecard stellt für die BUÖ ein geeignetes Instrument zur Umsetzung der wertorientierten Unternehmensstrategie dar. Herausforderungen bestehen insbesondere in der Auswahl geeigneter Kennzahlen sowie im laufenden Pflegeaufwand. Dennoch überwiegen die Vorteile einer transparenten, strategiekonformen Steuerung.